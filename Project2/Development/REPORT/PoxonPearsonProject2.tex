%%
%% Automatically generated file from DocOnce source
%% (https://github.com/hplgit/doconce/)
%%
%%


%-------------------- begin preamble ----------------------

\documentclass[%
oneside,                 % oneside: electronic viewing, twoside: printing
final,                   % draft: marks overfull hboxes, figures with paths
10pt]{article}

\listfiles               %  print all files needed to compile this document
\usepackage{mathtools}
\usepackage{relsize,makeidx,color,setspace,amsmath,amsfonts,amssymb}
\usepackage[table]{xcolor}
\usepackage{bm,ltablex,microtype}
\usepackage{float}
\usepackage[pdftex]{graphicx}
\usepackage{epstopdf}
\usepackage{verbatim}

\usepackage{fancyvrb} % packages needed for verbatim environments

\usepackage[T1]{fontenc}
%\usepackage[latin1]{inputenc}
\usepackage{ucs}
\usepackage[utf8x]{inputenc}
\usepackage[english]{babel}
\usepackage{lmodern}         % Latin Modern fonts derived from Computer Modern

% Hyperlinks in PDF:
\definecolor{linkcolor}{rgb}{0,0,0.4}
\usepackage{hyperref}
\hypersetup{
    breaklinks=true,
    colorlinks=true,
    linkcolor=linkcolor,
    urlcolor=linkcolor,
    citecolor=black,
    filecolor=black,
    %filecolor=blue,
    pdfmenubar=true,
    pdftoolbar=true,
    bookmarksdepth=3   % Uncomment (and tweak) for PDF bookmarks with more levels than the TOC
    }
%\hyperbaseurl{}   % hyperlinks are relative to this root

\setcounter{tocdepth}{2}  % levels in table of contents



% prevent orhpans and widows
\clubpenalty = 10000
\widowpenalty = 10000

% --- end of standard preamble for documents ---


% insert custom LaTeX commands...

\raggedbottom
\makeindex
\usepackage[totoc]{idxlayout}   % for index in the toc
\usepackage[nottoc]{tocbibind}  % for references/bibliography in the toc

%-------------------- end preamble ----------------------

\begin{document}

% matching end for #ifdef PREAMBLE

\newcommand{\exercisesection}[1]{\subsection*{#1}}


% ------------------- main content ----------------------



% ----------------- title -------------------------

\thispagestyle{empty}

\begin{center}
{\LARGE\bf
\begin{spacing}{1.25}
PHYS 905 - Project 2
\end{spacing}
}
\end{center}

% ----------------- author(s) -------------------------

\begin{center}
{\bf Terri Poxon-Pearson}
\end{center}

    
% ----------------- end author(s) -------------------------

% --- begin date ---
\begin{center}
February 27, 2017
\end{center}
% --- end date ---

\vspace{1cm}

INSERT ABSTRACT HERE

\tableofcontents
 
\section{Introduction}
In this project we will be solving Schroedinger's equation for the case of electrons in a 3D harmonic oscillator. First, we will first solve the simpler, non-interacting case with just one electron in the harmonic oscillator potential using Jacobi's method. Next, we will add a second electron with interacts via a repulsive, Coulomb potential.  This will involve a change into center of mass coordinates.  Both problems will be solved for l=0 case.  Additionally, we will explore our calculation's sensitivity to parameters such as the number of mesh points and maximum integration radius.

This report will begin with a brief introduction to the physical system we are studying in this project, as well as a description of how these equations are discretized.  Next I will describe the method used to solve this eigenvalue problem.  I will discuss the implementation of this algorithm, including a discussion of sensitivity to various integration variables and tests used to check the code's validity.  Finally, I will present the results of my calculations, followed by some conclusions and perspectives for future calculations.

\section{Methods}

\subsection{Physics of Single Electron in 3D Oscillator}
\subsection{Discretization of Schroedinger's Equation}
\subsection{Jacobi's Rotation Algorithm}
\subsection{Physics of Interacting Electrons in 3D Oscillator}


\section{Code and Implementation}

All of the programs, results, and benchmarks for this work can be found in my GIT repository ( https://github.com/poxonpea/PHYS905 ).  All codes for this project were written in FORTRAN.

\subsection{Implementing Single Electron Case}

\subsection{Implementing Interacting Electron Case}


\subsection{Tests of Code}


\section{Results and Discussion}


\subsection{Exploring Dependence of Integration Parameters}

\subsection{Convergence}


\subsection{Computational Speeds}


\section{Conclusions}



\section{Appendices}

\subsection{Appendix A}

Consider a basis of orthogonal basis vectors $\mathbf{v}_i$,
\[
\mathbf{v}_i = \begin{bmatrix} v_{i1} \\ \dots \\ \dots \\v_{in} \end{bmatrix}
\]

Orthogonality requires that 

\[
\mathbf{v}_j^T\mathbf{v}_i = \delta_{ij}.
\]

We can apply an orthogonal or unitary transformation such that

\[
\mathbf{w}_i=\mathbf{U}\mathbf{v}_i.
\]

Unitarity enforces that the product of a matrix with its conjugate transpose is the identity matrix.  Orthogonal matrices are a subset of real, unitary matrices.  This condition implies that the product of a matrix with its transpose is the identity matrix.  These two conditions can be expressed as

\[
\mathbf{U}^* \mathbf{U} = \mathbf{U} \mathbf{U}^* = \mathbb{I}
\]

\[
\mathbf{U}^T \mathbf{U} = \mathbf{U} \mathbf{U}^T = \mathbb{I}.
\]

If we now look at the product of our transformed matrix with its transpose, we find

\[
\mathbf{w}^T_j\mathbf{w}_i = (\mathbf{U}\mathbf{v}_i)^T(\mathbf{U}\mathbf{v}_j)= \mathbf{v}_i^T\mathbf{U}^T\mathbf{U}\mathbf{v}_j=\mathbf{v}_i^T\mathbf{U}\mathbf{U}^T\mathbf{v}_j =\mathbf{v}_j^T\mathbf{v}_i = \delta_{ij}.
\]

Therefore, the dot product is preserved.

\begin{comment}

\begin{figure}[H]\label{fig:compzoom}
  \centering
    \includegraphics[width=1.2\textwidth]{compzoom.eps}
    \caption{A zoomed in view of the convergence to the exact solution}
\end{figure}

\begin{center} 
\begin{tabular}{ |c|c|c|c| }
\hline
Size of Matrix ($10^n$) & General & Tailored & LU \\
\hline
1& 3.00 E -6 & 3.00 E -6 & 2.40 E -5\\ 
2 & 4.00 E -6 & 4.00 E -6 & 1.71 E -3 \\ 
3 & 3.90 E -5 & 1.90 E -5 & 1.93\\ 
4 & 3.79 E -4 & 2.09 E -4 & N/A\\ 
5 & 3.38 E -3 & 1.51 E -3  & N/A\\ 
6 & 2.87 E -2 & 1.53 E -2 & N/A\\ 
7 & 3.16 E -1 & 1.73 E -1& N/A\\ 
\hline
\end{tabular}
\label{table:test}
\end{center}

\end{comment}

\begin{thebibliography}{9}

\bibitem{LectureNotes} 
Hjorth-Jensen, Mortehn. 
Computational Physics, Lecture Notes Fall 2015. 
August 2015.

\bibitem{Chiarandini} 
Chiarandini, Marco. 
LU Factorization, Linear and Integer Programming. 
http://www.imada.sdu.dk/~marco/DM554/Slides/dm554-lu.pdf.

\bibitem{ClassNotes} 
Hjorth-Jensen, Mortehn. 
Introduction to Programming. 
Computational Physics.
https://compphysics.github.io/ComputationalPhysicsMSU/doc/pub/languages/pdf/languages-minted.pdf.

\end{thebibliography}



% ------------------- end of main content ---------------

\end{document}

